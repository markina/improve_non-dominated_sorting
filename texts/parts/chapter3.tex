\chapter{Реализация и экспериментальные исследования}
\label{chapter3}

В данной главе рассмотрены подробности реализации гибридного алгоритма, а также результаты экспериментов по сравнению
реализованного алгоритма с уже существующими.

Также в данной главе приведен пример задачи, использующей недоминирующую сортировку, которая использует гибридный
алгоритм, и проведено сравнение ее производительности с аналогичной реализацией задачи, использующей другой алгоритм
недоминирующей сортировки.

\section{Реализация гибридного алгоритма}

В данном разделе будут описаны подробности реализации гибридного алгоритма. Будут рассмотренны основные классы, а также
оптимизации, импользующиеся в данной реализации.

\subsection{Архитектура}

Тут будут описаны основные классы в реализации

\subsection{Оптимизации в алгоритме Роя}

Тут будут описаны детали реализации алгоритма Роя, которые ускоряют работу этого алгоритма, а следовательно и гибридного
алгоритма.

\subsubsection{Бинарный поиск}

В оригинальном алгоритме Роя используется линейный проход по массивам. %TODO: Дописать
Его можно заменить на бинарный поиск по элементам массива, немного ускорив его, хоть асимптотики алгоритма это и не
изменит

\subsection{Структуры данных}

Еще одна оптимизация, которая не меняет асимптотики алгоритма Роя, но добивается его ускорения -- использование
специальных структур данных, которые заменяют структуры, описанные в оригинальной статье.
%TODO: Дописать

\section{Сравнение с существующими алгоритмами на искусственно сгенерированных тестовых данных}

В данном разделе описывается относительная эффективность работы алгоритма. Пока что графиков нет, но скоро будут.

\section{Сравнение с существующими алгоритмами на практической задаче}

В данном разделе будет выбрана задача оптимизации, использующая наш гибридный алгоритм, и будет проведено сравнение
ее производительноси с такой же реализацией, но другим алгоритмом недоминирующей сортировки.


