\chapter{Практические исследования} 
\label{chapter3}

В данной главе рассмотрены эксперементальные результаты, полученные в течение работы.

Наша основная задача, как было описано в главе посвященной теоретическим исследованиям, заключается в том, чтобы быстро выбирать наилучший алгоритм, основываясь на входных данных.   
Для этого необходимо сначала собрать довольно много данных по времени их работы на разных входных данных. 

\subsection{Виды входных данных}
Рассмотрим какие именно входные данные будем рассматривать. 
\begin{itemize}
	\item Случайный точки в гиперкубе.
	\item Точки на одном ранге.
	\item Точки часто имеею одинаковый критерии. 
	\item Каждая точка имеет отличный от других ранг.
	\item Крайние случаи с шумами.
\end{itemize}

\subsection{Детали гибритизации}
Примитивный избиратель алгоритмов, который принимает решение в самом начале, основываясь на всех данных в целом.

Следующая идея сбора информации: сделаем дампы в момент каждого рекурсивного вызовов алгоритма Fast и сравним время сортировки этих подмножеств алгоритмами.


\section{Результаты экспериментов}



Тут требуется: 
\begin{itemize}
	\item Описать результат "кто в каких случаях лучше"
	\item Описать подсчет $\frac{T_\text{BOS} - T_\text{Fast}}{T_\text{max}}$
	\item Описать те эксперименты, которые еще будут проведены.
\end{itemize}

\subsection{Выводы}
По проведенным экспериментам можно сделать вывод, что надо продолжать исследовать зависимость количества ранов на скорость работы алгоритмов. 
