\startprefacepage

Множество известных и широко распространенных многокритериальных эволюционных алгоритмов используют процедуру
недоминирующей сортировки, или процедуру определения множества недоминирующих решений, которая может быть сведена
к недоминирующей сортировке. Примерами таких алгоритмов могут послужить NSGA-II \cite{NSGA-II}, PESA \cite{PESA},
PESA-II \cite{PESA-II}, SPEA2 \cite{SPEA2}, PAES \cite{PAES}, PDE \cite{PDE} и многие другие алгоритмы.
Вычислительная сложность одной итерации этих алгоритмов часто определяется сложностью процедуры недоминирующей
сортировки, следовательно, снижение сложности последней делает такие многокритериальные эволюционные алгоритмы
значительно быстрее.

Существуют разные алгоритмы недоминирующей сортировки, но эффективность их работы очень сильно отличается в зависимости
от данных. Этим можно воспользоваться и совместить идеи разных алгоритмов в одном, чтобы получить новый алгоритм,
сочетающий в себе преимущества существующих, избавившись при этом от их недостатков.

Цель данной работы -- сделать гибридный алгоритм недоминирующей сортировки, который будет использовать наиболее
подходящий алгоритм или переключаться между алгоритмами в ходе своей работы.

В Главе 1 данной работы представлен общий обзор работы.
В разделе 1.1 подробно рассмотрены определение недоминирующей сортировки и необходимые для этого понятия, а также
представлены примеры применения недоминирующей сортировки, подтверждающие актуальность данной работы.
В разделе 1.2 произведен обзор имеющихся результатов и подробно описаны лучшие из них.
В разделе 1.3 описаны недостатки существующих алгоритмов.
В разделе 1.4 сформулирована постановка задачи.

В Главе 2 представлены теоретические исследования по гибридизации алгоритмов недоминирующей сортировки.
В разделе 2.1 произведен анализ существующих алгоритмов и их сравнение.
В разделе 2.2 показаны основные проблемы, возникающие при гибридизации алгоритмов, а также предложены пути их решения.

В Главе 3 представлены практические исследования и их результаты.
В разделе 3.1 представлена общая архитектура программы и использованные оптимизации.
В разделе 3.2 приведены данные экспериментов по сравнению скорости работы гибридного алгоритма относительно старых.
В разделе 3.3 показано улучшение производительности практической задачи, использующей разработанный алгоритм для
недоминирующей сортировки.

В заключении подведены итоги работы, а также сказано, какие могут быть дальнейшие пути развития гибридных алгоритмов
недоминирующей сортировки.
