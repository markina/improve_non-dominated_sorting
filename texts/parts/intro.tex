\startprefacepage

Множество известных и широко распространенных многокритериальных эволюционных алгоритмов используют процедуру недоминирующей сортировки, или процедуру определения множества недоминирующих решений, которая может быть сведена к недоминирующей сортировке. Примерами таких алгоритмов могут послужить NSGA-II \cite{NSGA-II}, PESA \cite{PESA}, PESA-II \cite{PESA-II}, SPEA2 \cite{SPEA2}, PAES \cite{PAES}, PDE \cite{PDE} и многие другие алгоритмы. Вычислительная сложность одной итерации этих алгоритмов часто определяется сложностью процедуры недоминирующей сортировки, следовательно, снижение сложности последней делает такие многокритериальные эволюционные алгоритмы значительно быстрее.

Цель работы сделать гибридный алгоритм недоминирующей сортировки, который основываясь на входных данных будет использовать наиболее подходящий алгоритм или переключаться между алгоритмами в ходе своей работы.

В Главе 1 представлен общий обзор работы. В разделе 1.1 уточнены основные цели работы и рассмотрены термины и понятия, используемые в данной работе. В разделе 1.2 произведен анализ того, что требуется, что уже сделано и почему сделанного недостаточно. В разделе 1.3 представлена полная формулировка задачи, решаемой в данной работе.

В Главе 2 представлены основные теоретические исследования. 

В Главе 3 представлены практические исследования и их результаты.


