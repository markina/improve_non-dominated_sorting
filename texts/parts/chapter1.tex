
\chapter{Обзор работы}
\label{chapter1}

В этой главе представлен общий обзор работы: уточнены цели и объяснены термины и понятия, присутствующие в решении задачи. Также произведен обзор имеющихся результатов и сформулирована постановка задачи.

\section{Предметная область}
Предметная область:

\begin{itemize}
\item Алгоритмы недоминирующей сортировки.
\item Эволюционные алгоритмы.
\item Гибритизация алгоритмов.
\item Практические исследования.
\item Теоретические исследования.
\item Оценка времени работы алгоритмов.
\end{itemize}

\subsection{Применение}
Многокритериальные эволюционные алгоритмы работают с кандидатами решений, или особями. Каждый индивид оценивается функцией приспособленности $q=f(p)$, которая считается для каждой особи $p$ и является вектором из $M$ значений. 

Рассмотрим пример задачи, которая используют недомининирующую сортировку: $(\mu + \lambda)$ - эволюционная стратегия. Пусть есть начальное множество особей $S$, причем $|S| = \mu$. Далее по этому множеству генирируется новое множество особей $P$, где  $|P| = \lambda$. Следующим шагом необходимо выбрать результирующее множество $R$ из $S \cup P$, где $|S|+|P| \leq (\mu + \lambda)$, чтобы $|R|= \mu$. Требуется отобрать лучшие особи. Для этого необходимо произвести недоминирующую сортировку, тогда лучшими будут те, которые имеют наименьший ранг.

\subsection{Основные термины и понятия}

Далее мы рассмотрим основные определения, необходимые для понимания формулировки задачи. 

\subsubsection{Недоминирующая сортировка}
В $K$-мерном пространстве, точка $A = (a_1,...,a_M)$ доминирует точку $B = (b_1,...,b_M)$, когда для всех $1 \leq i \leq M$ выполняется неравенство $a_i\leq b_i$, и существует такое $j$, что $a_j < b_j$. Недоминирующая сортировка множества точек $S$ в $M$-мерном пространстве — это процедура, которая назначает всем точкам из $S$ ранг. Все точки, которые не доминируются ни одной точкой из $S$ имеют ранг нуль. Точка имеет ранг $i+1$, если максимальный ранг среди доминирующих её точек равен $i$.

\section{Анализ}

В данном разделе будет рассмотрено, что необходимо сделать, и проведено сравнение с тем, что уже есть.

\subsection{Имеющиеся результаты}
В работе Кунга и др. \cite{Kung} предлагается алгоритм определения множества недоминирующих точек, при этом его вычислительная сложность составляет $O(N log^{M-1} N)$, где $N$ — это число точек, а $M$ — размерность пространства. Этот алгоритм возможно использовать для выполнения недоминирующей сортировки. Сначала в множестве $S$ находятся множество точек с рангом $0$. Затем алгоритм Кунга запускается на оставшемся множестве точек, и получившемуся множеству точек присваивается ранг $1$. Процесс выполняется до тех пор, пока имеются точки, которым не присвоен ранг. Описанная процедура в худшем случае выполняется за $O(N^2 log^{M-1} N)$, если максимальный ранг точки равен $O(N)$.

Йенсен \cite{Jensen} впервые предложил алгоритм недоминирующей сортировки с вычислительной сложностью $O(N log^{M-1} N)$. Однако, как корректность, так и оценка сложности алгоритма доказывалась в предположении, что никакие две точки не имеют совпадающие значения ни в какой размерности. Устранить указанный недостаток оказалось достаточно трудной задачей — первой успешной попыткой сделать это, насколько известно исполнителю данной НИР, является работа Фортена и др. \cite{Forton}. Исправленный (или, согласно работе, «обобщенныий») алгоритм корректно работает во всех случаях, и во многих случаях его время работы составляет $O(N log^{M-1} N)$, но единственная оценка времени работы для худшего случая, доказанная в работе \cite{Jensen}, равна $O(N^2M)$. Наконец, в работе Буздалова и др. \cite{Buzdalov} предложены модификации алгоритма из работы \cite{Jensen}, которые позволили доказать в худшем случае также и оценку $O(N log^{M-1} N)$, не нарушая корректности работы алгоритма.

Большой интерес представляет алгоритм $Best~Order~Sort~(BOS)$ \cite{BOS}, который в отличии вышеупомянутых не использует метод разделяй и влавствуй. Его вычислительная сложность $O(MNlogM+MN^2)$. В лучшем случае алгоритм работает за $O(MNlogM)$, что лучше алгоритма предложенного Буздаловым и др. Однако в худшем случае его асимптотика совсем другая - $O(MN^2)$. Авторами алгоритма не было проведено более точных исследований по его времени работы. 

\section{Постановка задачи}

Задача содержит несколько пунктов: 
\begin{itemize}
	\item Выбрать ниаболее подходящие для гибритизации алгоритмы.
	\item Основываясь на практических экспериментах на разных видах входных данных, оценить время обработки каждым выбранным алгоритмом. 	
	\item Выдвинуть предположение о том, как и в какой момент менять стратегию сортировки. 
	\item Проверить предположение
	\item Сформулировать и реализовать гибридный алгоритм.
\end{itemize} 