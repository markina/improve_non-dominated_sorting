
\chapter{Обзор работы}
\label{chapter1}

В этой главе представлен общий обзор работы: уточнены цели и объяснены термины и понятия, присутствующие в решении
задачи. Также произведен обзор имеющихся результатов и сформулирована постановка задачи.

\section{Недоминирующая сортировка}

В данном разделе представлено определение недоминирующей сортировки и необходимые для ее понимания понятия.
Также рассмотрены случаи применения недоминирующей сортировки, которые обосновывают актуальность нового ускоренного
алгоритма недоминирующей сортировки.

\subsection{Определение}

Недоминирующая сортировка -- это процедура, которая ранжирует множество точек в многомерном пространсве $R^n$.
Если описывать неформально, то ее задача определить, какие точки \"лучше\" других. При этом допускается, что две точки 
могут быть одинаково \"хорошими\".

Для того, чтобы сформулировать определение недоменирующей сортировки, сначала надо определить, какие точки мы считаем 
\"лучше\" других. Для этого введем определение доминирования одной точки другой.

\textit{Определение.} В $M$-мерном пространстве, точка $A = (a_1,...,a_M)$ доминирует точку $B = (b_1,...,b_M)$
 тогда и только тогда, когда для всех $1 \leq i \leq M$ выполняется неравенство $a_i\leq b_i$, и существует такое $j$,
 что $a_j < b_j$.

\"Лучшими\" в данном контексте будут считаться точки, которые не доминируются ни одной другой точкой или, другми словами, 
лежащие на Парето-фронте. Однако часто бывают не только точки с парето-фронта, но и другие \"хорошие\" точки. Таким
образом мы приходим к определению процедуры недоминирующей сортировки.

\textit{Определение.} Недоминирующая сортировка множества точек $S$ в $M$-мерном пространстве — это процедура, которая 
назначает всем точкам из $S$ ранг. Все точки, которые не доминируются ни одной точкой из $S$ имеют ранг ноль. Точка 
имеет ранг $i+1$, если максимальный ранг среди доминирующих её точек равен $i$.

\subsection{Применение и актуальность}

Многокритериальные эволюционные алгоритмы работают с кандидатами решений, или особями. Каждый индивид оценивается
функцией приспособленности $q=f(p)$, которая считается для каждой особи $p$ и является вектором из $M$ значений.

Рассмотрим пример задачи, которая используют недомининирующую сортировку: $(\mu + \lambda)$ - эволюционная стратегия.
Пусть есть начальное множество особей $S$, причем $|S| = \mu$. Далее по этому множеству генерируется новое множество
особей $P$, где  $|P| = \lambda$. Следующим шагом необходимо выбрать результирующее множество $R$ из $S \cup P$,
где $|S|+|P| \leq (\mu + \lambda)$, чтобы $|R|= \mu$. Требуется отобрать лучшие особи. Для этого необходимо
произвести недоминирующую сортировку, тогда лучшими будут те, которые имеют наименьший ранг.

Таким образом, ускорив недомиминирующую сортировку, мы получит ускорение многокритериальных эвалючионных
алгоритмов, которые ее используют.

\section{Анализ существующих алгоритмов}

В данном разделе будут рассмотрена история развития алгоритмов недоминирующей сортировки. И особое внивание будет
уделено алгоритмам, которые применяются в гибридизации в данной работе.

\subsection{Наивные алгоритмы}

В работе Кунга и др. \cite{Kung} предлагается алгоритм определения множества недоминирующих точек, при этом его
вычислительная сложность составляет $O(N log^{M-1} N)$, где $N$ — это число точек, а $M$ — размерность пространства.
Этот алгоритм возможно использовать для выполнения недоминирующей сортировки. Сначала в множестве $S$ находятся
множество точек с рангом $0$. Затем алгоритм Кунга запускается на оставшемся множестве точек, и получившемуся
множеству точек присваивается ранг $1$. Процесс выполняется до тех пор, пока имеются точки, которым не присвоен ранг.
Описанная процедура в худшем случае выполняется за $O(N^2 log^{M-1} N)$, если максимальный ранг точки равен $O(N)$.

\subsection{Алгоритмы \"Разделяй и властвуй\"}

Йенсен \cite{Jensen} впервые предложил алгоритм недоминирующей сортировки с вычислительной сложностью
$O(N log^{M-1} N)$. Однако, как корректность, так и оценка сложности алгоритма доказывалась в предположении,
что никакие две точки не имеют совпадающие значения ни в какой размерности. Устранить указанный недостаток оказалось
достаточно трудной задачей — первой успешной попыткой сделать это, насколько известно исполнителю данной НИР,
является работа Фортена и др. \cite{Forton}. Исправленный (или, согласно работе, «обобщенный») алгоритм корректно
работает во всех случаях, и во многих случаях его время работы составляет $O(N log^{M-1} N)$, но единственная оценка
времени работы для худшего случая, доказанная в работе \cite{Jensen}, равна $O(N^2M)$. Наконец, в работе Буздалова
и др. \cite{Buzdalov} предложены модификации алгоритма из работы \cite{Jensen}, которые позволили доказать в худшем
случае также и оценку $O(N log^{M-1} N)$, не нарушая корректности работы алгоритма.

\subsection{Алгоритм Роя и др}

Большой интерес представляет алгоритм Роя $Best~Order~Sort~(BOS)$ \cite{Roy}, который в отличии вышеупомянутых не
использует метод разделяй и властвуй. Его вычислительная сложность $O(MNlogM+MN^2)$. В лучшем случае алгоритм
работает за $O(MNlogM)$, что лучше алгоритма предложенного Буздаловым и др. Однако в худшем случае его асимптотика
совсем другая - $O(MN^2)$. Авторами алгоритма не было проведено более точных исследований по его времени работы.


\subsection{Имеющиеся результаты}

\section{Постановка задачи}

Задача содержит несколько пунктов: 
\begin{itemize}
	\item Выбрать наиболее подходящие для гибридизации алгоритмы.
	\item Основываясь на практических экспериментах на разных видах входных данных, оценить время обработки каждым
	выбранным алгоритмом.
	\item Выдвинуть предположение о том, как и в какой момент менять стратегию сортировки. 
	\item Проверить предположение
	\item Сформулировать и реализовать гибридный алгоритм.
\end{itemize}
