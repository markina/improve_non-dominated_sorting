\documentclass{beamer}
\usetheme{CambridgeUS}

%\documentclass[handout]{beamer}
%\usetheme{Pittsburgh}
%\beamertemplatesolidbackgroundcolor{black!2}
%\setbeamertemplate{footline}[frame number]
%\usepackage{pgfpages}
%%\pgfpagesuselayout{4 on 1}[a4paper,border shrink=5mm,landscape]
%\pgfpagesuselayout{8 on 1}[a4paper,border shrink=5mm]

%%% PACKAGES
\usepackage[russian]{babel}
\usepackage[utf8]{inputenc}
\usepackage{amsmath}
\usepackage{amssymb}
\usepackage{tikz}
\usepackage{graphics}

%%% BEAMER SETTINGS
\setbeamertemplate{navigation symbols}{}
\setbeamertemplate{headline}{}

%%% TIKZ SETTINGS
\usetikzlibrary{fit}

%%% NEW COMMANDS
\def\pitem{\pause \item}

\DeclareGraphicsRule{.1}{mps}{*}{}
\DeclareGraphicsRule{.2}{mps}{*}{}
\DeclareGraphicsRule{.3}{mps}{*}{}
\DeclareGraphicsRule{.4}{mps}{*}{}
\DeclareGraphicsRule{.5}{mps}{*}{}
\DeclareGraphicsRule{.6}{mps}{*}{}

%\includeonlyframes{current} % leaves only the given frames

\title[Гибридный алгоритм недоминирующей сортировки]{Разработка гибридного алгоритма недоминирующей сортировки}
%\transduration{20}
\author[Маргарита Маркина]{Маргарита Маркина}
\institute[Университет ИТМО]{Национальный исследовательский университет информационных технологий, механики и оптики}
\date{}

\begin{document}

\begin{frame}
%\transduration{20}
\begin{center}
{\scriptsize Санкт-Петербургский национальный исследовательский университет \\ информационных технологий, механики и оптики}

\vspace{1cm}

{\scriptsize Факультет информационных технологий и программирования

Кафедра компьютерных технологий}

\vspace{1cm}

\vbox{\large\bfseries
Разработка гибридного алгоритма недоминирующей сортировки}

\vspace{1cm}

{\large Маркина Маргарита Анатольевна \\}
{\large Группа M3438}


\vspace{1cm}

{\large Научный руководитель: к.т.н. доцент кафедры КТ \\}
{\large М.~B.~Буздалов}


\end{center}
\end{frame}

%%%%%%%%%%%%%%%%%%%%%%%%%%%%%%%%%%%%%%%%%%%%%%%%%%%%%%%%%%%%%%%%%%%%%%%%%%%%%%%%%%%%%%%%%%%%%%%%%%%%%%%%%%%%%%%%%%%%%%%%
%\frame[label=title]{\titlepage}

\begin{frame}{Решаемая проблема}
%\transduration{20}
\begin{block}{Предметная область}
\begin{itemize}
\item aaa.
\item aaa.
\item aaa.
\item aaa.
\end{itemize}
\end{block}
\end{frame}

\begin{frame}{Решаемая проблема}
%\transduration{20}
\begin{block}{Введение}
\begin{itemize}
\item aaa.
\item aaa.
\item aaa.
\begin{center}
%\includegraphics*[height=3cm]{pic/illustrations.5}
\end{center}
\end{itemize}
\end{block}
\end{frame}

\begin{frame}{Решаемая проблема}
%\transduration{20}
\begin{block}{Цель исследования}
\begin{itemize}
\item aaa.
\item aaa.
\end{itemize}
\end{block}
\end{frame}

\begin{frame}{Постановка задачи}
%\transduration{20}
\begin{block}{Aaa $EA + RL$}
\begin{center}
%\includegraphics*[height=3cm]{pic/illustrations.6}
\end{center}
\begin{itemize}
\item $RL$: aaa.
\item $MO$: aaa.
\item $MO$: aaa.
\end{itemize}
\end{block}
\end{frame}

\begin{frame}{Решение}
%\transduration{20}
\begin{block}{Aaaaaa $O(N^2 \log N)$}
\begin{itemize}
\item Aaaa $N$ aaa
\item Aaaa
\item Aaaaaa
\end{itemize}
\end{block}
\end{frame}

\begin{frame}{Практические результаты}
%\transduration{20}
\begin{block}{Эксперименты}
Алгоритм запускался $100\,000$ раз при $N \in \{10, 20, 30, 40, 60\}$
\begin{table}[h]
\begin{center}
\begin{tabular}{r|c|c|c|c}
$N$ & Среднее & Дисперсия & $2eN \ln N$ & $\frac{16e^2}{7}N \ln N$ \\\hline
10 & 98.33 & 36.30 & 125.18 & 388.89 \\
20 & 269.90 & 78.11 & 325.73 & 1011.91 \\
30 & 476.38 & 121.37 & 554.72 & 1723.31 \\
40 & 704.24 & 163.87 & 802.19 & 2492.10 \\
50 & 950.93 & 208.66 & 1063.40 & 3303.56 \\
60 & 1209.98 & 251.52 & 1335.55 & 4149.03
\end{tabular}
\end{center}
\end{table}
\end{block}
\end{frame}

\begin{frame}{Практические результаты}
%\transduration{20}
\begin{block}{Эксперименты}

Aaaa aaaaaaa a a a aaaaa a aaaaaaaaa $N = 2$ и $N = 3$, aa aaaaaaa a $4 \leq N \leq 20$ и, aaaaa, aaaaaa $N$ тоже.

\begin{table}
\begin{center}
\begin{tabular}{|c|c|c|c|c|c|c|c|c|c|} \hline
N    & 4    & 5    & 6    & 7    & 8    & 9    & 10   & 11   & 12   \\\hline
diff & 0.03 & 0.10 & 0.15 & 0.18 & 0.20 & 0.21 & 0.23 & 0.24 & 0.25 \\\hline
\end{tabular}
\end{center}
\end{table}
\begin{table}
\begin{center}
\begin{tabular}{|c|c|c|c|c|c|c|c|c|} \hline
N    & 13   & 14   & 15   & 16   & 17   & 18   & 19   & 20   \\\hline
diff & 0.25 & 0.26 & 0.27 & 0.27 & 0.27 & 0.28 & 0.28 & 0.28 \\\hline
\end{tabular}
\end{center}
\end{table}
\end{block}
\end{frame}


\begin{frame}{Полученные результаты}
%\transduration{20}
\begin{block}{}
\begin{itemize}
\item Aaaaaaaa.
\item Aaaaaaaa.
\item Aaaaaaaa aaaaa.
\end{itemize}
\end{block}
\end{frame}

\begin{frame}{Дальнейшие действия}
%\transduration{20}
\begin{block}{}
\begin{itemize}
\item Aaaaa.
\item Aaaaaaa aaaa.
\end{itemize}
\end{block}
\end{frame}

\begin{frame}{}
\begin{center}
Спасибо за внимание!
\end{center}
\end{frame}

\begin{frame}{Дополнительные материалы}
%\transduration{20}
\begin{block}{Aaaaa}
Пример aaaa для $N = 3$
\begin{center}
%\includegraphics[height=5cm]{pic/illustrations.2}
\end{center}
\end{block}
\end{frame}

\end{document}